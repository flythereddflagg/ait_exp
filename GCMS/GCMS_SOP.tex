\documentclass[letterpaper,11pt]{article}

\usepackage[letterpaper, margin=0.75in]{geometry}
\usepackage{titlesec}
\usepackage[document]{ragged2e}
\usepackage{gensymb}
\usepackage{textcomp}
\usepackage{graphicx}
\usepackage{subfig}
\usepackage{float}
\usepackage{wrapfig}

\setlength{\parskip}{1em}

\DeclareGraphicsExtensions{.jpg,.JPG}

\newcommand*{\TitleFont}{
      \usefont{\encodingdefault}{\rmdefault}{b}{n}
      \fontsize{12}{20}
      \selectfont}

\newcommand\blankpage{%
    \null
    \thispagestyle{empty}%
    %\addtocounter{page}{-1}%
    \newpage}
    
\begin{document}

% generates the title
%\maketitle
\begin{center}
\TitleFont Gas Chromatograph Mass Spectrometer Standard Operating Procedures\newline
\TitleFont Last modified on: \today \hspace{1mm} by Mark Redd \newline
\end{center}

\titleformat{\section}
  {\normalfont\fontsize{12}{15}\bfseries}{\thesection}{1em}{}
 
% ~~~~ Does below section need to be included for GCMS???
\begin{itemize}
\item \textbf{NOTICE: Lab policy requires that any person performing 
    GCMS measurements must have done the following before performing any 
    experimental work:}
    
    \begin{itemize}
    \item \textbf{Complete pertinent laboratory safety training 
    \item Read this SOP in its entirety
    \item Become familiar with all the experimental steps outlined in this SOP
    \item Sign and Date the GCMS SOP Signatures Sheet}
    \end {itemize}
\end{itemize}

\renewcommand{\baselinestretch}{.75}\normalsize
\tableofcontents    % insert the table of contents
\renewcommand{\baselinestretch}{1.0}\normalsize
%\tableofcontents 


\section{Start up Procedures}
	\begin{itemize}
        \item Ensure that HiVac is not lower than $1.0*10^{-6}$ and not higher than $3.5*10^{-5}$
             
        \end{itemize}
        
\section{Tuning the GCMS}
	\subsection{Tune MSD should be done weekly}
    	%\begin{itemize}
    	%\item Input steps to do weekly tuning
    	%\end{itemize}
    \subsection{Quick Tune should be done daily}
    	\begin{itemize}
    	\item Save as pdf $\Longrightarrow$ TuneCreationDate
        	\begin{itemize}
        	\item Creation Date is listed, just copy and paste it in File Name
            	\begin{itemize}
            	\item Ex: Tune20151103091307
            \item Save on the Desktop in Folder ``MS Tune Reports''
            \item In pdf files, ensure Relative Abundance for Mass 502 is above 3\% and roughly the same as the day before
            	\end{itemize}
        	\end{itemize}
    	\end{itemize}
        
\section{Edit Method $\Longrightarrow$ This is where you tell the GCMS how to run (flow rates, temperatures, ramp rates, sample size, etc.)}
	\begin{itemize}
	\item This process is where the ``art'' of GCMS work comes in. Use p. 33-37 of black binder for MSD to determine how to change the method to get smoother/more resolved peaks.
    \item Click pencil icon underneath ``method''
    \item Everything on the left hand side should read ``Ready'' except GC Ready State
    \item Oven Icon
    	\begin{itemize}
    	\item Choose duration of temperature holding a ramp rate 
        \end{itemize}
    \item CFT Icon
    	\begin{itemize}
    	\item Choose Control method (flow rate or pressure) and specify
        \end{itemize}
    \item Inlets Icon
    	\begin{itemize}
    	\item Use the Back Inlet Tab
        \item Total Flow can \textbf{never} be above 1250 mL/min
        \item Split ratio can be adjusted as long as Total Flow is not too large
        \end{itemize}
    \item ALS Icon
    	\begin{itemize}
    	\item Choose injection volume
        	\begin{itemize}
        	\item The large the molecule, the smaller injection volume
        	\end{itemize}
		\begin{itemize}
		\item Solvent Washes
        	\begin{itemize}
        	\item Always want at least one wash for Solvent A (Acetone) both Pre and Post Injection to keep the needle clean. If necessary, use Solvent B (Methanol).
        	\end{itemize}
        \item Click \textbf{OK} until you reach \textbf{Save Method}
        \item Save method as \textbf{CompoundName.M} in Compound Folder
		\end{itemize}
    \end{itemize}
\end{itemize}

\section{Single Runs}
	\begin{itemize}
	\item Start the Run
    	\begin{itemize}
    	\item Click the big green arrow
        \item Change the operator intiials
        \item Ensure correct data path
        \item File Name
        	\begin{itemize}
        	\item If you have changed something in the method, include that in the file name
            	\begin{itemize}
            	\item Ex: NOV 2 HEXYLCYCLOHEXANE 5 	-- 190c, 25KMIN, SPLIT.D
            	\end{itemize}
        	\end{itemize}
    	\end{itemize}
        \item To Do the Next Run
        	\begin{itemize}
        	\item Repeat Procedure 5
            \item If you are changing the method in any way, you will also need to repeat Procedure 4
        	\end{itemize}
        \item Run Several \textbf{BLANK} runs between each sample run to flush out any residual sample (simply use an empty vial)
        	\begin{itemize}
        	\item When running blanks, label the runs around the sample runs:
            	\begin{itemize}
            	\item{} [Date] Blank 1aa (the first ``a'' means ``before the sample run'', the second ``a'' means ``the first blank done'')
                \item{} [Date] Blank lab
                \item{} [Date] [Sample] 1
                \item{} [Date] Blank 1ba (the ``b'' means ``after the sample run'')
                \item{} [Date] Blank 1bb
                \item{} [Date] Acetone 1 (to clean everything out)
                \item{} [Date] Blank 2aa
                \item{} [Date] Blank 2ab
                \item{} [Date] [Sample] 2...
            	\end{itemize}
        	\end{itemize}
	\end{itemize}

\section{Multiple Runs (Sequence)}
	\begin{itemize}
	\item Edit Sequence
    	\begin{itemize}
    	\item Click the pencil icon underneath ``Sequence''
        \item Keep data path the same as the method path (same folder)
        \item Choose Type (Sample or Blank), Vial position, and Sample Name
        \item Method should be spelled the exact same as the \textbf{.M} file used (CompoundName)      
    	\end{itemize}
    \item Running Sequences
    	\begin{itemize}
    	\item Include the desired number of runs in the Edit Sequence step
        \item To run, click the running man underneath ``Sequence'' instead of the big green arrow, then click ``Run Sequence''
    	\end{itemize}
	\end{itemize}
    
\section{Analysis}
	\begin{itemize}
	\item Purity
    	\begin{itemize}
    	\item Tmplibrp.txt file includes acetone in purity, so use rteres.txt file to compute actual purity
        	\begin{itemize}
        	\item Add up the corrected peak areas excluding acetone and divide each peak area by the total
        	\end{itemize}
    	\end{itemize}
    \item Compound identification
    	\begin{itemize}
    	\item Open AMDIS from the Desktop
        	\begin{itemize}
        	\item Open the file you wish to analyze
           	\item Click Run near the top of the window
            \item In the top graph, zoom in on a peak
            \item Click on the top of the peak
            \item Analyze $\Longrightarrow$ Go to NIST MS Program
            \item Right center graph compares MS data (red) to known compound MS (blue)
            	\begin{itemize}
            	\item Use tabs below to select different comparisons 
            	\end{itemize}
            \item Lower left list gives compounds that are similar to collected MS
            	\begin{itemize}
            	\item Prob. Column gives probability of a match
            	\end{itemize}
            \item On the bar graph, the farther to the left the red bar is, the better the math
            \item Bottom graph gives structure of selected compound
        	\end{itemize}
    	\end{itemize}
	\end{itemize}

\end{document}