\documentclass{article}
\usepackage[margin=0.5in]{geometry} % set margins and general document geometries
\usepackage[utf8x]{inputenc}
\usepackage[T1]{fontenc} 			% looks better than default font encoding
\usepackage{lmodern} 				% need a newer font to work with T1 font encoding
\usepackage{graphicx} 				% for including images
\usepackage{enumitem} 				% for adjusting lists
\usepackage{setspace}				% for double spacing	
\usepackage{pgffor}					% needed for the for-loop
\usepackage{tabu}

%%%%% PANDOC REQUIRED %%%%%%%%%%%%%%%%%
%% pandoc-tablenos: required package
\usepackage{caption}
\usepackage{longtable}
\usepackage{booktabs,siunitx}

% DO NOT USE THE hyperref PACKAGE!
% \usepackage{hyperref} 
% for citing using the hyperlink command
\newcommand{\href}{\cite}
\newcommand{\hyperlink}{\ref}
\newcommand{\hypertarget}[1]{\ignorespaces}
\def\tightlist{}

%% pandoc-tablenos: environment to disable table caption prefixes
\makeatletter
\newcounter{tableno}
\newenvironment{tablenos:no-prefix-table-caption}{
  \caption@ifcompatibility{}{
	\let\oldthetable\thetable
	\let\oldtheHtable\theHtable
	\renewcommand{\thetable}{tableno:\thetableno}
	\renewcommand{\theHtable}{tableno:\thetableno}
	\stepcounter{tableno}
	\captionsetup{labelformat=empty}
  }
}{
  \caption@ifcompatibility{}{
	\captionsetup{labelformat=default}
	\let\thetable\oldthetable
	\let\theHtable\oldtheHtable
	\addtocounter{table}{-1}
  }
}
\makeatother

%% pandoc-fignos: environment to disable figure caption prefixes
\makeatletter
\newcounter{figno}
\newenvironment{fignos:no-prefix-figure-caption}{
  \caption@ifcompatibility{}{
	\let\oldthefigure\thefigure
	\let\oldtheHfigure\theHfigure
	\renewcommand{\thefigure}{figno:\thefigno}
	\renewcommand{\theHfigure}{figno:\thefigno}
	\stepcounter{figno}
	\captionsetup{labelformat=empty}
  }
}{
  \caption@ifcompatibility{}{
	\captionsetup{labelformat=default}
	\let\thefigure\oldthefigure
	\let\theHfigure\oldtheHfigure
	\addtocounter{figure}{-1}
  }
}
    \makeatother
%%%%% END PANDOC REQUIRED %%%%%%%%%%%%%

\DeclareGraphicsExtensions{.jpg,.png}
\graphicspath{{./media/}}

\title{\textbf{AIT Measurement Standard Operating Procedure}}
\date{\textbf{Last modified on: \today \hspace{1mm} by Mark Redd}}

\begin{document}
\maketitle

\begin{itemize}
\item \textbf{NOTICE: Lab policy requires that any person performing 
    AIT measurements must have done the following before performing any 
    experimental work:}
    
    \begin{itemize}
    \item \textbf{Complete pertinent laboratory safety training 
    \item Read this SOP in its entirety
    \item Become familiar with all the experimental steps and safety protocols 
        outlined in this SOP (This includes everything in Sections 
        \ref{sec:experiments}, \ref{sec:data}, \ref{sec:spills} and 
        \ref{sec:e_shtdn})
    \item Sign and date this page}
    \end {itemize}
\end{itemize}

By signing this, I assert that I have completed the above steps and have 
sufficient competence to carry out these experiments safely. 
\begin{center}
\begin{tabu} to 0.9\textwidth {X[c]|X[c]|X[c]}
 Full Name (Please Print) & Signature & Date  \\ 
 \hline
 \vspace{5mm} & & \\ \hline 
 \vspace{5mm} & & \\ \hline 
 \vspace{5mm} & & \\ \hline 
 \vspace{5mm} & & \\ \hline 
 \vspace{5mm} & & \\ \hline 
 \vspace{5mm} & & \\ \hline 
 \vspace{5mm} & & \\ \hline 
 \vspace{5mm} & & \\ \hline 
 \vspace{5mm} & & \\ \hline 
 \vspace{5mm} & & \\ \hline 

 
\end{tabu}
\end{center}

\newpage
% \renewcommand{\baselinestretch}{.4}\normalsize
\tableofcontents    % insert the table of contents
% \renewcommand{\baselinestretch}{1.0}\normalsize

\newpage

% PARITAL DOCUMENT BODY SECTION
% If you would like to only compile one file from your document,
% comment out the section below and uncomment the line below 
% and specify the section by the prefix you
% have specified in the content folder (e.g. 15-Section-Name.md would
% be section15).

% \input{section15}

% DOCUMENT BODY SECTION
% This is the code for the entire document based on the .tex source
% files in the ./tex/ folder.

\maketitle

\foreach \i in {10,...,99} {%
    \edef\FileName{./tex/section\i}%     The % here are necessary to eliminate any
    \IfFileExists{\FileName}{%  spurious spaces that may get inserted
       \input{\FileName}%       at these points
    }
}

\end{document}
