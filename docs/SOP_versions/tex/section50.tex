\hypertarget{spill-clean1.up}{%
\section{Spill Clean1.up}\label{spill-clean1.up}}

In the event of any spill, appropriate PPE specified in the
corresponding SDS should be used in clean1.up. Always check the SDS for
special considerations when cleaning up any compound.

\hypertarget{liquids}{%
\subsection{Liquids}\label{liquids}}

In the event of a small spill (i.e.~less than 100 ml), the following
protocol should be followed:

\begin{itemize}
\tightlist
\item
  If the spill occurs in or out of the hood, use absorbent clay that can
  be found under the counter west of the sink to soak up the bulk of the
  liquid and wipe up the rest with a paper towel

  \begin{itemize}
  \tightlist
  \item
    Dispose of the clay, any disposable gloves and towels in the solid
    waste container
  \item
    In the event of a large spill (i.e.~greater than 100 ml), the
    following protocol should be followed:
  \end{itemize}
\item
  If the spill occurs in the hood, use absorbent clay that can be found
  under the counter on the left1.hand side of the lab sink to soak up
  the bulk of the liquid and wipe up the rest with a paper towel

  \begin{itemize}
  \tightlist
  \item
    Dispose of the clay, any disposable gloves and towels in the solid
    waste container
  \item
    If the spill occurs outside the hood or the spill is particularly
    large (e.g.~an entire bottle of a flammable material breaks)
    \textbf{perform the Emergency Shutdown Procedure} (Section
    refsec:e\_shtdn), evacuate the lab and call: BYU Risk Management and
    Safety - (801) 422-4468
  \end{itemize}
\item
  Spills involving compounds that are particularly toxic or unstable
  should always be considered large spills
\end{itemize}

\hypertarget{solids}{%
\subsection{Solids}\label{solids}}

We will generally work with organic solids that readily dissolve in
simple organic solvents (e.g.~acetone). Researchers must always check
chemical compatibility with solutes and solvents before dissolving any
compound.

\begin{itemize}
\item
  Small amounts of organic solids may be dissolved in a small amount of
  solvent and put in organic liquid waste
\item
  Larger amounts of solids should be transferred to solid waste and the
  residue should be dissolved in solvent and discarded in liquid waste
\end{itemize}
