\hypertarget{data-extraction}{%
\section{Data Extraction}\label{data-extraction}}

During experiments data are being recorded on the lab computer, the data
logger and the camera. Both the camera and the data logger on the TADA
have SD cards with a 32 GB storage capacity that allows multiple runs to
be recorded without extraction. The following policies are in place to
ensure ease of use, efficiency, and avoid common mistakes.

\hypertarget{general-policies}{%
\subsection{General Policies}\label{general-policies}}

\begin{itemize}
\item
  All data, including video and raw temperature data should be extracted
  at least \emph{daily}
\item
  Video data should be extracted and properly renamed as often as
  possible (i.e.~between every run or every other run) to ensure the
  correct filenames are assigned to their corresponding video files
\item
  The datalogger data is there as a redundant backup to the UI data in
  case of data loss. Therefore it will be mainly archived and used only
  when the original data cannot be found
\item
  After processing, all data should be organized according to the
  following conventions:

  \begin{itemize}
  \tightlist
  \item
    Path: \texttt{\$PREFIX/compound\_name/filename.ext}

    \begin{itemize}
    \tightlist
    \item
      \texttt{\$PREFIX} for video data:
      \texttt{smb://pgl6ed.byu.edu/aitra/video}
    \item
      \texttt{\$PREFIX} for all other data:
      \texttt{/home/aitra/Documents/data}
    \end{itemize}
  \item
    All experiments should have a unique filename associated with them
    according to convention
  \item
    All data from the experiment run should have the same filename but
    different extensions
  \item
    The corresponding data from the datalogger will be archived and
    accessed as needed

    \begin{itemize}
    \tightlist
    \item
      Path: \texttt{/home/aitra/Documents/Data}
    \end{itemize}
  \item
    When processing is finished, all experiments should have the
    following 3 files with the same name preceding them

    \begin{itemize}
    \tightlist
    \item
      A .png/jpg) file (for temperature data with graphs and analysis)
    \item
      A .csv file (TADA-generated)
    \item
      A .avi/.mp4 file stored on the DIPPR Legacy Server (this video
      file will have a different path and extension but the same
      filename)
    \end{itemize}
  \end{itemize}
\item
  File naming convention:

  \begin{itemize}
  \item
    Filenames will be organized by the following values in order
    separated by underscores (`\texttt{\_}')

    \begin{enumerate}
    \def\labelenumi{\arabic{enumi}.}
    \tightlist
    \item
      Compound name
    \item
      Date of experiment with the format ``YYMMDD''
    \item
      Time of day that data collection began for that run using a 24
      hour clock format ``hhmm''
    \item
      Sample size in microliters (for liquids) or milligrams(for solids
      and gases)
    \item
      Test temperature in degrees Celsius (rounded to the nearest
      integer)
    \end{enumerate}
  \item
    For example: The file name of an AIT experiment where 100
    microliters of hexane were tested at 450 degrees Celsius on March
    19, 2013 at 4:25 pm would be:
    \texttt{hexane\_130319\_1625\_100\_450.csv}
  \item
    The corresponding video file would be named:
    \texttt{hexane\_130319\_1625\_100\_450.MP4"}
  \end{itemize}
\end{itemize}

\hypertarget{video-extraction}{%
\subsection{Video Extraction}\label{video-extraction}}

The following procedure is necessary only if you have not or cannot
extract and delete video via WiFi (See Section ???).

\begin{enumerate}
\def\labelenumi{\arabic{enumi}.}
\tightlist
\item
  Connect the camera to the computer via a micro USB cable (See Figure
  reffig:cam\_diag)
\item
  Press the ``info/wireless'' button on the camera to connect the camera
  to the computer
\item
  A new icon should appear allowing you to access the SD card as if it
  were a USB drive.
\item
  Video files should be copied to the DIPPR Legacy Server (a.k.a. The
  Properties of Gases and Liquids 6th Edition Server) and organized as
  explained above

  \begin{itemize}
  \tightlist
  \item
    Path: \texttt{smb://pgl6ed.byu.edu/aitra/video}
  \item
    Domain: \texttt{dipprleg}
  \item
    Username: \texttt{aitra}
  \item
    Password: \texttt{hotflame16}
  \end{itemize}
\item
  Once you have ensured all video data have been have been properly
  saved in the appropriate data folder, delete all files from the camera
\end{enumerate}

If you have trouble with the above method, you may remove the SD card
and copy the video to the server using a similar method as below.

\hypertarget{datalogger-extraction}{%
\subsection{Datalogger Extraction}\label{datalogger-extraction}}

To extract data from the datalogger:

\begin{enumerate}
\def\labelenumi{\arabic{enumi}.}
\item
  Unplug the TADA from the computer
\item
  Pull out the SD card from the data logger and use the USB SD card
  adapter to copy the \texttt{DATALOG.CSV} file into the ``raw\_data''
  path and rename it to the original filename with the date tagged on in
  ``YYMMMDD'' format (e.g.~\texttt{DATALOG\_130319.CSV})
\item
  Path: \texttt{/home/aitra/Documents/data/raw\_data/}
\item
  Once you have ensured the data log file has been copied and renamed
  successfully, delete the \texttt{DATALOG.CSV} file on the SD card (the
  SD card should be empty)
\item
  Close all windows with the USB SD card adapter open (i.e.~Windows
  Explorer etc.)
\item
  Pull out the SD card without ejecting the unit from the computer
\end{enumerate}
