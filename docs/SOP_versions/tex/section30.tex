\hypertarget{experimental-design}{%
\section{Experimental Design}\label{experimental-design}}

When performing AIT measurements researchers will have relatively small
sample sizes to work with therefore following guidelines apply to
minimize the amount of compound needed to effectively ascertain the AIT.

\hypertarget{sample-size-procedure}{%
\subsection{Sample Size Procedure}\label{sample-size-procedure}}

There are 5 standard sample sizes specified by the ASTM method they are
as follows:

\begin{itemize}
\tightlist
\item
  For solids: 50, 70, 100, 150, 200, and 250 milligrams (mg)
\item
  For liquids: 50, 70, 100, 150, 200, and 250 microliters (μL)
\item
  For gases: 50, 70, 100, 150, 200, and 250 milligrams (mg)
\end{itemize}

Acceptable errors for these sample sizes is +/- 10 mg/μL.

For any compound measurement, there must be a minimum of 3 sample sizes
tested. The following steps must be observed:

\begin{enumerate}
\def\labelenumi{\arabic{enumi}.}
\item
  Start with as sample size of 100 mg/μL and find the minimum AIT
  (explained below).
\item
  Always do 150 mg/μL next and find the minimum AIT for that sample size
\item
  Compare the minimum AIT's from the first two sample sizes:

  \begin{itemize}
  \item
    If 100 mg/μL gives a lower AIT, do 50 mg/μL next.
  \item
    If 150 mg/μL gives a lower AIT, do 250 mg/μL next.
  \end{itemize}
\item
  Find the minimum AIT for the third sample size.
\item
  Compare the minima from all three experiments and determine if further
  tests are needed:

  \begin{enumerate}
  \def\labelenumii{\arabic{enumii}.}
  \item
    Find the \% error between the lowest and the other two using the
    following formula
    \(\%Error_i = \frac{|AIT_{lowest} - AIT_{i}|}{AIT_{lowest}} * 100 \%\)
    where \(AIT_{lowest}\) is the lowest AIT between the three and
    \(AIT_i\) is the AIT of one of the other two. You should get two
    error values.
  \item
    If both error values are \(\leq 2.0\%\) then report the lowest value
  \item
    If either error values are \(> 2.0\%\) then further tests are needed

    \begin{itemize}
    \item
      If you did the 50 mg/μL sample size, find the minimum AIT with a
      sample size of 70 mg/μL
    \item
      If you did the 250 mg/μL sample size, find the minimum AIT s with
      a sample size of 200 mg/μL
    \end{itemize}
  \end{enumerate}
\item
  Report the minimum AIT found between all of the sample sizes
\end{enumerate}

\hypertarget{finding-the-minimum-ait}{%
\subsection{Finding the minimum AIT}\label{finding-the-minimum-ait}}

The methodology for finding a minimum AIT for a particular sample size
involves a bisection method described as follows:

\begin{enumerate}
\def\labelenumi{\arabic{enumi}.}
\tightlist
\item
  Do the first experiment at a reasonable temperature.

  \begin{itemize}
  \tightlist
  \item
    Usually this involves choosing a starting temperature based on a
    predicted value or looking at family plots of the compound.
  \end{itemize}
\item
  Bracket the minimum AIT by finding a temperature where ignition was
  observed and one lower temperature where no ignition was observed.

  \begin{itemize}
  \tightlist
  \item
    If the initial temperature produced a hot-flame ignition, decrease
    the temperature.
  \item
    If a cold-flame ignition was observed, decrease the temperature to
    find a non-ignition then increase to find a hot-flame ignition.
  \item
    Begin by changing the temperature by at least 10 K. If this new
    temperature does not successfully bracket the minimum AIT, double
    the temperature change until a bracket is found.
  \item
    E.g. Suppose you have no ignition at 450 K. Increase to 460 K
    produces no ignition. Increase to 480 K produces no ignition.
    Increase to 520 K produces a hot-flame ignition. Your bracket is
    \(480 K < AIT \leq 520 K\).
  \end{itemize}
\item
  Once the minimum has been bracketed, bisect the temperature space
  until the bracket size is \(\leq 3.0K\).

  \begin{itemize}
  \tightlist
  \item
    E.g. if you have an ignition at 450 K, a non-ignition at 430 K and
    no measurements between the two, your next experiment should be at
    440 K. Suppose that experiment ignites. The next temperature should
    be at 435 K. If that does not ignite, the next temperature should be
    at 437.5 K. If that ignites, then you have successfully bracketed
    the minimum between 437.5 K and 440 K with a bracket size less than
    3 K.
  \end{itemize}
\item
  The last step is to confirm the minimum. To ensure the minimum has
  been found, at least 3 experiments must confirm that there is no
  ignition observed within \(3.0 K\) below the minimum ignition
  temperature. If lower ignition temperatures are observed, continue
  doing confirmation experiments until there are at least 3
  non-ignitions within \(3.0K\) below the lowest observed ignition value
  \textbf{that were NOT part of the bisection process}. If cold
  ignitions are observed, they may be considered the same as
  non-ignitions for the purposes of finding the hot-flame AIT.
\item
  The lowest temperature where a hot-ignition was observed is the
  reported AIT for that sample size.
\end{enumerate}
