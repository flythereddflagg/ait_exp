% ------------------------ Last Modified on: 08/16/16 -------------------------

\documentclass[letterpaper,11pt]{article}

\usepackage[letterpaper, margin=0.5in]{geometry}
\usepackage{titlesec}
\usepackage[document]{ragged2e}
\usepackage{gensymb}

\newcommand*{\TitleFont}{
      \usefont{\encodingdefault}{\rmdefault}{b}{n}%
      \fontsize{12}{20}
      \selectfont}

\begin{document}

% generates the title
%\maketitle
\begin{center}
\TitleFont AIT Measurement Apparatus Standard Operating Procedure\newline
\TitleFont Last modified on: \today \hspace{1mm} by Mark Redd \newline
\end{center}


\titleformat{\section}
  {\normalfont\fontsize{12}{15}\bfseries}{\thesection}{1em}{}

% insert the table of contents
%\tableofcontents

\begin{itemize}
\item \textbf{NOTICE: Lab policy requires that any person performing 
    experiments with the AIT Measurement Apparatus must have done the 
    following before performing any experimental work:}
    
    \begin{itemize}
    \item \textbf{Complete pertinent laboratory safety training which includes 
        becoming familiar with all the steps outlined in this SOP}
    \item \textbf{Read this SOP in its entirety}
    \item \textbf{Sign and Date the AMA SOP Signatures Sheet}
    \end {itemize}
\end{itemize}

\section{Preparation}
    \begin{itemize}
    \item Flask and Lid
        
        \begin{itemize}
        \item \textbf{Latex or nitrile gloves and safety glasses are 
            required while preparing the flask}
        \item Use a 500 mL, round bottom, long-neck flask
        \item If dirty, wash out the flask using soap and water and dry as much 
            as possible (see Section 3: Clean-up/Shutdown)
            
                \begin{itemize}
                \item Any leftover water will boil away when the furnace heats 
                up and before any measurements are taken
                \end{itemize}
                
        \item Wrap entire flask in aluminum foil with thermocouples at the 
            bottom, side and top of the round part of the flask (thermocouples 
            should be touching the glass directly) % gloves?
            
                \begin{itemize}
                \item Start by getting a long strip of aluminum foil (8" wide or 
                    so) and wrapping it around the middle of the flask
                \item Poke thermocouple 3 through the foil near the bottom so 
                    the bead sits at the bottom of the flask and then wrap the 
                    foil around the bottom
                \item Slide thermocouple 2 down to the middle of the flask 
                    between the flask and foil and start to wrap the foil up 
                    the flask
                \item Place thermocouple 1 at the top of the bulb of the flask 
                    and wrap the rest of the foil up around the top
                \item Wrap additional foil around the neck of the flask to cover 
                    it completely and secure flask in lid assembly
                \item Make a "donut" of foil that will rest up against the 
                bottom of the lid assembly
                \end{itemize}
                
        \item Loosen the nut on top of the lid assembly and slide the 
            corresponding half of the ceramic part of the lid assembly out
        \item Place the flask in the ceramic part of the lid assembly with the 
            lip of the flask fitting into the groove of the ceramic
        \item Slide the loose half of the ceramic back in to be snug 
            around the flask neck and tighten the nut on the top to hold it
            in position
                
                \begin{itemize}
                \item The two halves nearest to the top of the assembly should 
                    meet or very nearly meet; if they don't then some 
                    foil should be removed from the neck of the flask
                \item Use a circular spring to help hold the halves together
                \end{itemize}
                
        \item Slide the foil "donut" up so it is flush against the ceramic and 
            basically seals the opening
        \item Carefully turn the flask/lid assembly over making sure the flask 
            doesn't fall out
            
                \begin{itemize}
                \item \textbf{Do this over a table or close to a level surface 
                    to avoid accidental breaking of the flask}
                \item The flask will fit into the lid assembly somewhat loosely, 
                    but it shouldn't fall out
                \item If the flask falls out, remove it and add more foil 
                around the neck
                \end{itemize}
                
        \item Guide the thermocouple wires in the gap between the two ceramic 
            halves so they are out of the way when the flask/lid assembly is 
            inserted into the furnace
        \item Place the prepared flask/lid assembly into the furnace
        \end{itemize}
            
    \item Furnace
        
        \begin{itemize}
        \item Power on furnace and set temperature for initial measurement
        
                \begin{itemize}
                \item To change the set point, press the up or down arrows until 
                    the desired temperature is reached
                \end{itemize}
                
        \item Insert flask interior thermocouple (\#4) carefully down the flask 
            neck, making sure it goes straight in and the bead doesn't get 
            caught anywhere
                
                \begin{itemize}
                \item The bead of thermocouple \#4 should be suspended in the 
                    approximate middle of the flask, not be touching any part
                \item Use the bracket on one of the two handles on top of the 
                    lid to secure the thermocouple in place
                \end{itemize}
                
        \item Connect the thermocouples to the DAQ
        \end{itemize}
        
    \end{itemize}

\section{Measurement and Data Collection}

    \begin{itemize}
    \item Start up computer and log on
    
        \begin{itemize}
        \item Username: McKay
        \item Password: asdfghjkl (Home row on a QWERTY keyboard)
        \end{itemize}
        
    \item Open "AIT Data Collection 2016.vi" (Shortcut Located on Desktop)
    \item Press the "run" button to start the program
        
    \item Enter the filename in the textbox or use the browse dialog to save to 
        the right path
        
        \begin{itemize}
        \item Path: $"C:\backslash AIT\backslash compound\_name
            \backslash filename .txt"$
        
        \item Filename naming convention: 
            
            \begin{itemize}
            \item Filenames will be organized by the following values in order 
                separated by underscores ("\_")
                
                    \begin{itemize}
                    \item Compound name
                    \item Sample size in microliters
                    \item Temperature in degrees Celsius
                    \item Date of experiment with the format "YYMMDD"
                    \item Time of day that data collection began for that run
                        using a 24 hour clock format "hhmm"
                    \end{itemize}
                    
            \item For example: The filename of an AIT experiment where 100 
                microliters of hexane were tested at 450 \degree C on March 19, 
                2013 at 4:25 pm would be: "hexane\_100\_450\_130319\_1625.txt"
            \end{itemize}    
                
        \item Make sure to press enter to save the filename in the LabVIEW 
            program
        \end{itemize}
    
    \item \textbf{Safety: Ensure you are using proper PPE and hazards in 
            the lab environment are minimized before continuing}
            
            \begin{itemize}
            \item Ensure your workspace, the area around the computer and 
                both hoods are free of clutter, tripping hazards or any object 
                which could present a hazard to you or anyone else in the lab
            \item Refer to the SDS for the chemical you are working 
                with when determining appropriate PPE
                
                \begin{itemize}
                \item NOTE: Some SDS's will recommend using a face shield in 
                    addition to splash goggles when handing their respective 
                    chemicals. In our lab we will use ventilation hoods which, 
                    when used properly, serve as better protection than face 
                    shields. Therefore, any time an SDS recommends using a 
                    face shield you may safety ignore that recommendation 
                    provided you are using the hood properly by positioning the 
                    sash between your face and the work being performed in 
                    the hood.
                \end{itemize}
                
            \item Appropriate hand protection (e.g. nitrile gloves) and 
                splash goggles are required when handling chemicals
            \item Lab coats are recommended but not required when handling 
                chemicals
            \item All chemical handling (except for injection into the furnace) 
                should be done in a hood other than the hood containing the 
                furnace to avoid a potential fire hazard 
            \end{itemize}
    
    \item Measure out sample
    
        \begin{itemize}
        \item Draw sample amount into a right-angle syringe
        \item Sample size:
        
            \begin{itemize}
            \item Initially use a sample size of 100 microliters
            \item Once AIT is measured for 100 microliters, go to 150 
                microliters
            \item If the AIT decreases for 150 microliters, go to 200/250 
                microliters
            \item If the AIT increases for 150 microliters, go to 50 microliters
            \end{itemize}
        
        \end{itemize}
        
    \item Ensure the lab is sufficiently dark to see any flame from the mirror
        on top of the furnace
        \begin{itemize}
        \item Use the lights in the hoods for preparation before your run
        \end{itemize}
    \item Set a timer for 10 minutes but don't start it yet
    \item Center the end of the syringe in the hole at the top of the furnace 
        for injection, ensuring that the sample will go straight down and not 
        hit the sides of the flask
    \item Depress the pedal marked "D" to initiate data collection
    \item Turn off the lights
    \item Introduce your sample about 5 seconds after initiating data 
        collection
        
        \begin{itemize}
        \item Immediately withdraw the syringe and place it in the adjacent hood
        \item Begin the 10 minute timer
        \end{itemize}    
            
    \item Watch the mirror above the furnace for any flame/glow for 10 minutes
        
        \begin{itemize}
        \item If a flame or glow is observed, document it (color, size, 
            brightness, sound) and then continue data collection for 1 minute
            after the flame or glow has disappeared, then terminate data 
            collection by pressing the "D" pedal again
                
                \begin{itemize}
                \item If the flame is bright yellow/orange, this is the 
                    hot-flame auto-ignition and the temperature should be 
                    decreased for the next test
                \item If the flame is faint and blueish, this is the cool-flame 
                    auto-ignition and the temperature should be increased for 
                    the next test
                \item \textbf{The reported AIT is the minimum temperature at 
                    which hot-flame ignition occurs}
                \item If no flame or glow if observed by the 10 minute mark, 
                    increase the temperature for the next measurement
                \item \textit{The bracket size goal for AIT measurement is $\pm$
                3 \degree C}
                \end{itemize}
        
        \end{itemize}
        
    \item Prepare for the next measurement
        
        \begin{itemize}
        \item Set furnace to next temperature
        \item Clean out the flask between measurements by blowing hot air into 
            the flask for 5 minutes using the heat gun
        \item Wait a minimum of 10 minutes between measurements for the furnace 
            to equilibrate at the next temperature (5 min w/heat gun, 5 min 
            to equilibrate)
        \end{itemize}    
    
    \item Start this procedure over from the third step (measuring out a sample)
    \end{itemize}

\section{Clean-up/Shutdown}

    \begin{itemize}
    \item Under normal use, disposable gloves may be thrown into the normal 
        trash receptacle instead of solid chemical waste
    \item The furnace may be too hot to open for several hours
    \item Once the furnace is cool, remove flask/lid assembly
    \item Remove flask from lid assembly and remove the aluminum foil
    \item Wash out flask with soap and water (scrubbing stains if necessary) and 
        place on drying rack
        \begin{itemize}
        \item For difficult stains, soak the flask inside with soapy water for 
            24 hours or longer if needed
        \item For best results, flasks must be as clean as possible
        \end{itemize}
    \item Do not rinse out needles
    \end{itemize}

\section{Spill Clean-up}
    \begin{itemize}
    \item In the event of any spill, appropriate PPE specified in the 
        corresponding SDS should be used in clean-up
    \item In the event of a small spill (i.e. less than 100 ml), the following 
        protocol should be followed:
        
        \begin{itemize}
        \item If the spill occurs in or out of the hood, use absorbent clay that
            can be found in the lab to soak up the bulk of the liquid and wipe 
            up the rest with a paper towel
        \item Dispose of the clay, any disposable gloves and towels in the solid 
            waste container
        \end{itemize}
    
    \item In the event of a large spill (i.e. greater than 100 ml), the 
        following protocol should be followed:
        
        \begin{itemize}
        \item If the spill occurs in the hood, use absorbent clay that can be 
            found in the lab to soak up the bulk of the liquid and wipe up the 
            rest with a paper towel
        \item Dispose of the clay, any disposable gloves and towels in the solid 
            waste container
        \item If the spill occurs outside the hood or the spill is particularly 
            large (e.g. an entire bottle of a flammable material breaks) 
            \textbf{perform the Emergency Shutdown Procedure (Section 5), 
            evacuate the lab and call: BYU Risk Management and Safety - 
            (801)-422-4468} 
        \end{itemize}
    
    \end{itemize}

\section{Emergency Shutdown}

    \begin{itemize}
    \item In the event of an emergency do the following:
        
        \begin{itemize}
        \item Power off the furnace
        \item Unplug the furnace
        \item Close all programs and shutdown the computer
        \end{itemize}
    
    \item If an emergency requires you to evacuate the lab do only the first 
        2 steps
    \end{itemize}

\end{document}