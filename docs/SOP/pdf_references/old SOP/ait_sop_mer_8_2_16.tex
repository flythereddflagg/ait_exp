% ------------------------ Last Modified on: 08/02/16 -------------------------

\documentclass[letterpaper,11pt]{article}

\usepackage[letterpaper, margin=0.5in]{geometry}
\usepackage{titlesec}
\usepackage[document]{ragged2e}
\usepackage{gensymb}

\newcommand*{\TitleFont}{
      \usefont{\encodingdefault}{\rmdefault}{b}{n}%
      \fontsize{12}{20}
      \selectfont}

\begin{document}

% generates the title
%\maketitle
\begin{center}
\TitleFont AIT Measurement Apparatus Standard Operating Procedure\newline
\TitleFont Last modified on: \today \hspace{1mm} by Mark Redd \newline
\end{center}


\titleformat{\section}
  {\normalfont\fontsize{12}{15}\bfseries}{\thesection}{1em}{}

% insert the table of contents
%\tableofcontents

\section{Preparation}
    \begin{itemize}
    \item Flask and Lid
        
        \begin{itemize}
        \item Use a 500 mL, round bottom, long-neck flask
        \item If dirty, wash out the flask using soap and water and dry as much 
            as possible
            
                \begin{itemize}
                \item Any leftover water will boil away when the furnace heats 
                up and before any measurements are taken
                \end{itemize}
                
        \item Wrap entire flask in aluminum foil with thermocouples at the 
            bottom, side and top of the round part of the flask (thermocouples 
            should be touching the glass directly) % gloves?
            
                \begin{itemize}
                \item \textbf{Latex or nitrile gloves and safety glasses are 
                    required while preparing the flask}
                \item Start by getting a long strip of aluminum foil (8" wide or 
                    so) and wrapping it around the middle of the flask
                \item Poke thermocouple 3 through the foil near the bottom so 
                    the bead sits at the bottom of the flask and then wrap the 
                    foil around the bottom
                \item Slide thermocouple 2 down to the middle of the flask 
                    between the flask and foil and start to wrap the foil up 
                    the flask
                \item Place thermocouple 1 at the top of the bulb of the flask 
                    and wrap the rest of the foil up around the top
                \item Wrap additional foil around the neck of the flask to cover 
                    it completely and secure flask in lid assembly
                \item Make a "donut" of foil that will rest up against the 
                bottom of the lid assembly
                \end{itemize}
                
        \item Loosen the nut on top of the lid assembly and slide the 
            corresponding half of the ceramic part of the lid assembly out
        \item Place the flask in the ceramic part of the lid assembly with the 
            lip of the flask fitting into the groove of the ceramic
        \item Slide the loose half of the ceramic back in to be snug 
            around the flask neck and tighten the nut on the top to hold it
            in position
                
                \begin{itemize}
                \item The two halves nearest to the top of the assembly should 
                    meet or very nearly meet; if they don't then some 
                    foil should be removed from the neck of the flask
                \item Use a circular spring to help hold the halves together
                \end{itemize}
                
        \item Slide the foil "donut" up so it is flush against the ceramic and 
            basically seals the opening
        \item Carefully turn the flask/lid assembly over making sure the flask 
            doesn't fall out
            
                \begin{itemize}
                \item The flask will fit into the lid assembly somewhat loosely, 
                    but it shouldn't fall out
                \item If the flask falls out, remove it and add more foil 
                around the neck
                \end{itemize}
                
        \item Guide the thermocouple wires in the gap between the two ceramic 
            halves so they are out of the way when the flask/lid assembly is 
            inserted into the furnace
        \item Place the prepared flask/lid assembly into the furnace
        \end{itemize}
        
    \pagebreak    
    
    \item Furnace
        
        \begin{itemize}
        \item Power on furnace and set temperature for initial measurement
        
                \begin{itemize}
                \item To change the set point, press the up or down arrows until 
                    the desired temperature is reached
                \end{itemize}
                
        \item Insert flask interior thermocouple (\#4) carefully down the flask 
            neck, making sure it goes straight in and the bead doesn't get 
            caught anywhere
                
                \begin{itemize}
                \item The bead of thermocouple \#4 should be suspended in the 
                    approximate middle of the flask, not be touching any part
                \item Use the bracket on one of the two handles on top of the 
                    lid to secure the thermocouple in place
                \end{itemize}
                
        \item Connect the thermocouples to the DAQ
        \end{itemize}
        
    \end{itemize}

\section{Measurement and Data Collection}

    \begin{itemize}
    \item Start up computer and log on
    
        \begin{itemize}
        \item Username: McKay
        \item Password: asdfghjkl (Home row on a QWERTY keyboard)
        \end{itemize}
        
    \item Open "AIT Data Collection 2016.vi" (Shortcut Located on Desktop)
    \item Press the "run" button to start the program
        
    \item Enter the filename in the textbox or use the browse dialog to save to 
        the right path
        
        \begin{itemize}
        \item Path: $"C:\backslash AIT\backslash compound\_name
            \backslash filename .txt"$
        
        \item Filename naming convention: 
            
            \begin{itemize}
            \item Filenames will be organized by the following values in order 
                separated by underscores ("\_")
                
                    \begin{itemize}
                    \item Compound name
                    \item Sample size in microliters
                    \item Temperature in degrees Celsius
                    \item Date of experiment with the format "YYMMDD"
                    \end{itemize}
                    
            \item For example: The filename of an AIT experiment where 100 
                microliters of hexane were tested at 450 C on March 19, 2013 
                would be: "hexane\_100\_450\_130319.txt"
            \end{itemize}    
                
        \item Make sure to press enter to save the filename in the LabVIEW 
            program
        \end{itemize}

    \item Measure out sample
    
        \begin{itemize}
        \item \textbf{Appropriate hand protection (e.g. nitrile gloves) and 
            splash goggles are required when handling chemicals}
        \item Draw sample amount into a right-angle syringe
        \item Sample size:
        
            \begin{itemize}
            \item Initially use a sample size of 100 microliters
            \item Once AIT is measured for 100 microliters, go to 150 
                microliters
            \item If the AIT decreases for 150 microliters, go to 200/250 
                microliters
            \item If the AIT increases for 150 microliters, go to 50 microliters
            \end{itemize}
        
        \end{itemize}
        
    \item Ensure the lab is sufficiently dark to see any flame from the mirror
        on top of the furnace
    \item Set a timer for 10 minutes but don't start it yet
    \item Depress the pedal marked "D" to initiate data collection
    \item Introduce your sample about 3 to 5 seconds after initiating data 
        collection
        
        \begin{itemize}
        \item Immediately begin the 10 minute timer
        \end{itemize}    
            
    \item Watch the mirror above the furnace for any flame/glow for 10 minutes
        
        \begin{itemize}
        \item If a flame or glow is observed, document it (color, size, 
            brightness, sound) and then continue data collection for 1 minute
            after the flame or glow has disappeared, then terminate data 
            collection by pressing the "D" pedal again
                
                \begin{itemize}
                \item If the flame is bright yellow/orange, this is the 
                    hot-flame auto-ignition and the temperature should be 
                    decreased for the next test
                \item If the flame is faint and blueish, this is the cool-flame 
                    auto-ignition and the temperature should be increased for 
                    the next test
                \item \textbf{The reported AIT is the minimum temperature at 
                    which hot-flame ignition occurs}
                \item If no flame or glow if observed by the 10 minute mark, 
                    increase the temperature for the next measurement
                \item \textit{The bracket size goal for AIT measurement is $\pm$
                3 \degree C}
                \end{itemize}
        
        \end{itemize}
        
    \item Prepare for the next measurement
        
        \begin{itemize}
        \item Set furnace to next temperature
        \item Clean out the flask between measurements by blowing hot air into the 
            flask for 5 minutes using the heat gun
        \item Wait a minimum of 10 minutes between measurements for the furnace to 
            equilibrate at the next temperature (5 min w/heat gun, 5 min 
            to equilibrate)
        \end{itemize}    
    
    \item Start this procedure over from the third step (measuring out a sample)
    \end{itemize}

\section{Clean-up/Shutdown}

    \begin{itemize}
    \item The furnace may be too hot to open for several hours
    \item Once the furnace is cool, remove flask/lid assembly
    \item Remove flask from lid assembly and remove the aluminum foil
    \item Wash out flask with soap and water (scrubbing stains if necessary) and 
        place on drying rack
    \item Do not rinse out needles
    \end{itemize}

\section{Emergency Shutdown}

    \begin{itemize}
    \item In the event of an emergency do the following:
        
        \begin{itemize}
        \item Power off the furnace
        \item Unplug the furnace
        \item Close all programs and shutdown the computer
        \end{itemize}
    
    \item If an emergency requires you to evacuate the lab do only the first 
        2 steps
    \end{itemize}
    
\end{document}