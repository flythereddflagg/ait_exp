
\documentclass[letterpaper,11pt]{article}

\usepackage[letterpaper, margin=0.75in]{geometry}
\usepackage{titlesec}
\usepackage[document]{ragged2e}
\usepackage{gensymb}
\usepackage{textcomp}
\usepackage{graphicx}
\usepackage{float}
\usepackage{wrapfig}


\setlength{\parskip}{1em}

\DeclareGraphicsExtensions{.jpg,.JPG}

\newcommand*{\TitleFont}{
      \usefont{\encodingdefault}{\rmdefault}{b}{n}
      \fontsize{12}{20}
      \selectfont}

\newcommand\blankpage{%
    \null
    \thispagestyle{empty}%
    %\addtocounter{page}{-1}%
    \newpage}
      
\begin{document}

% generates the title
%\maketitle
\begin{center}
\TitleFont AIT Measurement Standard Operating Procedure\newline
\TitleFont Last modified on: \today \hspace{1mm} by Mark Redd \newline
\end{center}

\titleformat{\section}
  {\normalfont\fontsize{12}{15}\bfseries}{\thesection}{1em}{}

\begin{itemize}
\item \textbf{NOTICE: Lab policy requires that any person performing 
    AIT measurements must have done the following before performing any 
    experimental work:}
    
    \begin{itemize}
    \item \textbf{Complete pertinent laboratory safety training 
    \item Read this SOP in its entirety
    \item Become familiar with all the experimental steps outlined in this SOP
    \item Sign and Date the AIT SOP Signatures Sheet}
    \end {itemize}
\end{itemize}
%\renewcommand{\baselinestretch}{.75}\normalsize
\tableofcontents    % insert the table of contents
%\renewcommand{\baselinestretch}{1.0}\normalsize
%\tableofcontents 
    
\newpage
\section{Measurement and Data Collection}
This section enumerates the procedure for measuring AIT. Researchers 
should follow these procedures every day and for every experiment performed to 
ensure consistent results. The first priority should always be safety. 
Therefore, if any step of this process is found to be unsafe or pose an 
unacceptable risk it should be changed. Furthermore, changes should be made if 
any step of the process violates the ASTM E659 Method to conform to the 
requirements of the method.
    
    \subsection{Startup}
    
	Before you work with the furnace you should ensure 
            that it is at a safe temperature %%the next few steps explain how to do this
    
    \begin{enumerate}
    \item Ensure the lid is off the pressure vessel and the vessel is being 
        vented by the snorkel
        \begin{itemize}
        \item Under normal operation, the vessel should be vented with the 
            snorkel any time the vessel is open
        \item The only exception to this rule is when the experimental setup has
            been shut down for an extended period of time for maintenance 
            purposes
        \end{itemize}
    \item Start up computer and log on
        \begin{itemize}
      
        \item Use your CAEDM account to log in
            \begin{itemize}
            \item You should be able to access all the needed tools and 
                programs from your account
            \item If you cannot access a program or file from your account, let 
                Mark know and he will give you administrator access as needed
            \end{itemize}
        
        \item You may need to specify the domain you are logging into. If that 
            is the case enter your CAEDM credentials in as follows:
            \begin{itemize}
            \item Username: CAEDM\_AD\textbackslash
            \textit{your\_caedm\_username}
            \item Password: \textit{your\_caedm\_password}
            \end{itemize}
        
        %\item Using the administrator account:
        %    \begin{itemize}
        %    \item Username: CHEME-R-2014-25\textbackslash AIT\_LAB
        %    \item Password: Explosionsinthesky16 (Explosions in the sky 16)
        %    \end{itemize}
        
        \end{itemize}
    
    \item Ensure a compatible SD card is inserted securely into the TA-DA 
        datalogger
        
\begin{figure}[H]
\centering
\includegraphics[width=.2\textwidth]{sdcard_reader.jpg}
\caption{SD card reader}
\label{fig:SDcard_reader}
\end{figure}
        
    \item Ensure the thermocouples are connected to the TA-DA %%Connectors? Picture?
    \item Connect the TA-DA to the lab computer via the USB cable mounted 
        under the edge of the hood
        
\begin{figure}[H]
\centering
\includegraphics[width=.5\textwidth]{TA-DA.jpg}
\caption{connect TA-DA}
\label{fig:TA-DA}
\end{figure}
        
    \item Plug in the 24 volt power supply to the TA-DA
    \item Open the TA-DA user interface program 
            \begin{itemize}
            \item Path: \textit{C:\textbackslash Users\textbackslash  
                Public\textbackslash Public Documents\textbackslash AIT\textbackslash 
                ait\_exp\textbackslash TADA\textbackslash TADA\_UI.py}
            \item You may wish to make a shortcut to this location and put it on 
                your CAEDM desktop.
            \item Upon opening the program, a yellow LED in the TA-DA should 
                begin flashing (See Figure \ref{fig:tada_leds}) %Check to see that you are getting data
            \end{itemize}

    \item Return to the computer and press the "Sync Time" button on the bottom 
        left corner of the TADA\_UI
        window to synchronize the Arduino clock to the computer time
        \begin{itemize}
        \item This must be done at least once every work day
        \end{itemize}

\begin{figure}[H]
\centering
\includegraphics[width=.75\textwidth]{led_red_yellow.jpg}
\caption{Control LEDs inside the TA-DA}
\label{fig:tada_leds}
\end{figure}
        
    \item Use the TADA\_UI program to track the internal temperature of the 
        flask as it heats up initially
    \item Check the temperature of the furnace to ensure safe handling for 
        startup
        \begin{itemize}
        \item The temperature should be close to ambient lab temperature
        \item Do not begin experiments until the internal temperature is below
            $100^\circ C$
        \end{itemize}
    \item Ensure the ARIA lead screws are set such that the ARIA is in the 
        shutdown position %Expand on this/ pictures?
    \item Ensure the three molex cables are securely plugged in to the ARIA and
        will not come out while the ARIA is moving
        
\begin{figure}[H]
\centering\includegraphics[width=.5\textwidth]{Molex_Cables_ARIA2.jpg}
\caption{Molex cables secured and out of the way}
\label{fig:Molex_Cables_ARIA}
\end{figure}
        
    \item Align the ARIA so that the needle or funnel is aligned to the center
        of the injection hole in the top of the furnace
        \begin{itemize}
        \item You will need to install the appropriate setup for the compound
            you intend to test that day
        \item After alignment, remove any setup materials used to reach 
            alignment
        \end{itemize}
    \item Plug in the ARIA control to the 5 volt power supply and wait for 
        initial setup sequence to complete (The two button lights on the ARIA
        control will come on when the sequence is complete)
    \item Test the placement of the ARIA using a blank sample
         \begin{itemize}
         \item For liquids/gases, place the empty syringe in the ARIA, lined up 
            over the flask, and run the Liquid program.
         \item For solids, place the ring stand in the ARIA, then secure the 
            funnel in the ring stand, then run the solid program
         \item If the syringe or the funnel do not go directly into the flask, 
            adjust the placement and retest until properly positioned
         \end{itemize}
    \item Ensure the furnace is plugged in
    \item Power on furnace and set furnace temperature between 20 - 30 degrees 
        above your initial target flask temperature 
    \item Reduce the set point temperature when the internal flask temperature 
        exceeds your initial target temperature by 5 - 10 degrees 
        \begin{itemize}
        \item Wait a minimum of 90 minutes before changing the temperature
        \item When powered on initially, the furnace may take up to 2 hours or   
            more to reach a desired temperature and thermally equilibrate
        \end{itemize}
    
    \item While the furnace heats initially, prepare the tablet and the camera
        \begin{itemize}
        \item Power on the tablet and connect it to the camera's Wi-Fi (See 
            Section \ref{sec:cam_tab} on how to do this)
        \item Open the Capture app to view the camera's view finder
        \item Mount the camera on the tripod using the plastic quick-release
            mount on top of the tripod
        \item Using the camera's view finder on the tablet, adjust the
            position of the camera, tripod, and mirror on the furnace to 
            align the camera's view to see through the sight glass directly down
            the center of the flask
            \begin{itemize}
            \item The camera should be positioned looking into the sight glass
                so that the hole in the furnace \textit{and} the mirror 
                are visible 
            \item For the best view, the camera should be as close to the 
                sight glass as the tripod will allow
            \end{itemize}
        
        \item Gently tighten the knobs on the tripod to fix the camera in place
        \end{itemize}
    
    \item Once the target temperature is reached, allow 30 minutes for thermal 
        equilibration, then begin experiments
    \end{enumerate}
    
\subsection{Experimental}
This section outlines the steps for experimental runs. Each experiment should be
performed following these steps exactly (insofar as that is possible). Doing so
will ensure consistent results with the lowest uncertainty possible.
    \begin{enumerate}
  
    \item In the TADA\_UI program, press the "Choose Target File" button and 
        choose where to save your file
        \begin{itemize}
        \item Save all temperature data files in comma separated values (.csv) 
            format
        \item Path: \textit{C:\textbackslash Users\textbackslash 
            Public\textbackslash Documents\textbackslash AIT\textbackslash 
            data\textbackslash compound\_name\textbackslash filename.csv}
        \item File naming convention: 
            \begin{itemize}
            \item Filenames will be organized by the following values in order 
                separated by underscores ("\_")
                    \begin{itemize}
                    \item Compound name
                    \item Phase of the compound ('g' for gases, 'l' for liquids,
                        's' for solids)
                    \item Date of experiment with the format "YYMMDD"
                    \item Time of day that data collection began for that run
                        using a 24 hour clock format "hhmm"
                    \item Sample size in microliters (for liquids) or milligrams
                        (for solids and gases)
                    \item Test temperature in degrees Celsius (rounded to the 
                        nearest integer)
                    \end{itemize}
                    
            \item For example: The file name of an AIT experiment where 100 
                microliters of liquid hexane were tested at 450 \degree C on 
                March 19, 2013 at 4:25 pm would be: \newline
                "hexane\_l\_130319\_1625\_100\_450\_.csv"
            \end{itemize}    
        \item This action will reset the TA-DA for the next measurement        
        \end{itemize}
    
    \item \textbf{Safety: Ensure you are using proper PPE and have minimized 
        hazards in the lab environment before continuing}
            
            \begin{itemize}
            \item Ensure your workspace, the area around the computer and 
                both hoods are free of clutter, tripping hazards or any object 
                which could present a hazard to you or anyone else in the lab
            \item Appropriate PPE (e.g. nitrile gloves and splash goggles) are 
                required when handling chemicals
            \item Refer to the SDS for the chemical you are working 
                with when determining appropriate PPE
                \begin{itemize}
                \item NOTE: Some SDS's will recommend using a face shield in 
                    addition to splash goggles when handing their respective 
                    chemicals. In our lab we will use ventilation hoods which, 
                    when used properly, serve as better protection than face 
                    shields. Therefore, any time an SDS recommends using a 
                    face shield you may safely ignore that recommendation 
                    provided you are using the hood properly by positioning the 
                    sash between your face and the work being performed in 
                    the hood.
                \end{itemize}
                
            \item Unless an SDS states otherwise, lab coats are recommended but 
                not required when handling chemicals
            \item All chemical handling (except for injection into the furnace) 
                should be done in the left hood to avoid a potential fire hazard 
            \end{itemize}
    
    \item Measure out sample
        \begin{itemize}
        \item Liquids
            \begin{itemize}
            \item Draw sample amount into a right-angle syringe
                \begin{itemize}
                \item This should not exceed about 250 microliters
                \end{itemize}
            \end{itemize}
        
        \item Solids
            \begin{itemize}
            \item Tare the lab scale with the weigh boat and measure out sample
                \begin{itemize}
                \item This should not exceed 250 mg
                \end{itemize}
            \end{itemize}
            
        \item Gases
            \begin{itemize}
            \item Draw sample amount into a right-angle syringe  
                \begin{itemize}
                \item This should not exceed 250 microliters
                \end{itemize}
            \end{itemize}
            
        \end{itemize}

     \item Secure the sample to the ARIA
		\begin{itemize}
		\item Liquid/Gases Sample
			\begin{itemize}
			\item Place the syringe securely into the syringe holder on the 
                ARIA, making sure the tip of the syringe 
                needle is aligned down the center of the hole in the furnace
			\end{itemize}
			
		\item Solid Sample
			\begin{itemize}
			\item Carefully insert weigh boat onto holder on the ARIA servo
			\end{itemize}
			
		\end{itemize}
			
	\item Remove gloves before proceeding
	
	\item Place lid on pressure vessel and secure in place with the 
        clamps and cable
		\begin{itemize}
        \item When placing the lid, hold the lid directly above the vessel and 
            carefully lower it straight down on the vessel to avoid hitting the
            ARIA with the lid
        \item Line up the two marks on the lid with the corresponding marks 
            on the vessel
        \item Ensure the lid is centered on the vessel by running your fingers
            around the edge to ensure the edge of the vessel and the lid are
            flush 
		\item Hand tighten all the pressure vessel clamps on the lip of the lid so 
            the slack is taken out
        \item Tighten the clamps in opposing pairs following the numbering on 
			the back of each clamp (see Figure \ref{fig:Top_of_lid}) using a torque wrench, tightening each clamp until the torque reads 60 inch pounds
 
\begin{figure}[H]
\centering
\includegraphics[width=.35\textwidth]{Top_of_lid.jpg}
\caption{Clamp numbering}
\label{fig:Top_of_lid}
\end{figure}
         
        \item Tighten each clamp again with the torque wrench, this time by going around
			the circle, ensuring that torque is 60 inch pounds on each.
        \item Loop the safety cable through both lid handles and through the handles on
			both sides of the vessel and then up through the lid handles and secure the two ends 
			together (See Figure \ref{fig:Cable})
           
\begin{figure}[H]
\centering
\includegraphics[width=.35\textwidth]{Cable.png}
\caption{secure cable}
\label{fig:Cable}
\end{figure}

		\end{itemize}
		
	\item Pressurize vessel
		\begin{itemize}
		\item Open the TADA\_UI in the ait\_exp folder. The absolute pressure in the vessel
			is read at the bottom of the window highlighted in blue
        \item Ensure the ball valve is closed
        \item Fully open the rotometer on the exhaust of the pressure vessel
        \item Slowly open the cylinder valve all the way and then turn back one 
			quarter turn
        \item Once the regulators are pressurized, slowly open the ball valve
            \begin{itemize}
            \item Note: If a loud, high pitched noise is heard, there is a 
                leak. Immediately close the ball valve and allow the vessel to 
                fully vent to ambient pressure and check all seals
            \item Ensure that there are no leaks
            \end{itemize}
        \item Slowly close the rotometer until the air flow reads 0.4 SCFM 
            (the lowest graduation on the rotometer)
		\item Adjust pressure in vessel using low pressure regulator until the    
            absolute pressure reading in the TADA\_UI is highlighted green indicating that the 
			pressure in the vessel is 1 atm (760 Torr)
		\end{itemize} 

	\item Ensure the vessel is sufficiently dark to see any 
        flame from the mirror on top of the furnace
	
	\item Begin data collection 
		\begin{itemize}
		\item In the TA-DA UI program, press the Enter key
		\item A red LED in the TA-DA should begin blinking
		\item The TA-DA UI will keep track of the elapsed time since data 
          collection began at the bottom of the window. This may be used
          to time the experiment
		\end{itemize}
	
    \item Press the red button on the tablet 
        screen to start recording
		
	\item Press the LED button on the ARIA control box that corresponds to the 
        physical state of the sample
		(green for solid, blue for liquid) to initiate ARIA sample injection.
            
    \item Watch the TA-DA UI for a large temperature spike for 10 minutes
        \begin{itemize}
        \item If a temperature spike is observed, allow enough time for the
            temperature to return to a steady state before stopping the camera 
            and terminating temperature data collection
        
        \item The experiment ends when one of the following criteria is met:
            \begin{itemize} 
            \item An ignition event is observed (i.e. a temperature spike) 
                and the temperature returns to steady state
            \item 10 minutes pass with no ignition event observed
            \end{itemize}
        \item If the UI is used to keep track of time and no flame is observed, 
            continue collecting temperature data until 600 seconds have passed 
            since injection
        \end{itemize}
        
    \item Review the camera footage, looking for a flame corresponding to the 
        temperature spike
    
    \item Record pertinent data and observations in the lab book and the 
        TA-DA UI
        \begin{itemize}
        \item The following data must be present on the same row, in the 
            following order:
            \begin{itemize}
            \item Time of day that data collection began for that run
            \item Compound name 
            \item The lot number and/or sample number of the 
                container (This only needs to be recorded in the lab book  and 
                only once for every compound container)
            \item Phase of the compound upon injection ('g' for gases, 'l' for 
                liquids, 's' for solids)
            \item Sample size in microliters (for liquids) or milligrams
                (for solids and gases)
            \item Set-point temperature of the furnace
            \item Test temperature in degrees Celsius (rounded to the 
                nearest integer)
                \begin{itemize}
                \item This should be the internal flask temperature 
                    (Thermocouple 4) prior to injection
                \end{itemize}
            \item Indicate whether an ignition event was observed (i.e. Did 
                the camera see a flame?)
            \item Indicate whether a hot flame or cold flame was observed (if 
                applicable)
                \begin{itemize}
                \item If the flame is bright yellow/orange, this is considered a 
                    hot-flame autoignition
                \item If the flame is faint and blueish, this is considered a 
                    cool-flame autoignition
                \end{itemize}
            \item Indicate if any sound was heard upon ignition (if applicable)
            \end{itemize}
        \item If any item is not applicable write down N/A in its place
        \item If any item is unknown, leave a blank until it can be determined
        \item Optionally, leave any pertinent comments about the experiment
            next to or directly under this row of data
        \item Record the same data in the corresponding fields in the TA-DA UI
            \textbf{before} terminating temperature data collection
        \end{itemize}
        
    \item After the experiment ends, terminate data collection
        \begin{itemize}
        \item Press the Enter key  again to stop data 
            collection (the red light on the TA-DA should stop blinking)
        \item Press the red button on the tablet screen to stop recording
        \item Turn the lights back on
        \item Video recording may be stopped as soon as a flame or glow
            disappears
        \end{itemize}
 
    \item Wait about 20 minutes to allow for the pressure 
        vessel to be purged of the combustion products
    \item Extract, save and appropriately rename the video data between 
            experiments
    \item During the purge time, set furnace to next temperature
        \begin{itemize}
            \item When changing temperature, always approach your target
                temperature from 5 - 10 degrees Celsius above and then 
                descending slowly to your target temperature
        \end{itemize}	
	\item After the purge time as ended, turn off the inlet air flow and wait 
        for pressure vessel to \textit{fully} depressurize (outflow rotometer reads zero)
	\item Remove the pressure vessel lid 
		\begin{itemize}
		\item Loosen the clamps in the same way that they were tightened, this time 
			using a regular wrench instead of a torque wrench
		\item Remove the safety cable \textit{after} you remove the clamps
		\item Lift off the lid and\textit{immediately} place snorkel inside the 
		pressure vessel approximately 6" over the furnace 
		\end{itemize}
		
    \item Clean out the flask between measurements by blowing hot air into
            the flask for 5 minutes using the heat gun
            \begin{itemize}
            \item The heat gun should \textbf{only} be plugged in to the outlet 
                when in use
            \end{itemize}
            
        \item Wait a minimum of 30 minutes to reach the new temperature and 
            allow the furnace and flask to thermally equilibrate
    \item Start this procedure over from step 1 (choose the target file)
    \end{enumerate}

	
\subsection{Shutdown}
    \begin{itemize}
    \item The following should be done before leaving the lab at the end of 
        every work day or any time the setup is not in use:
        \begin{itemize}
        \item Power off the furnace
        \item Unplug TA-DA from the computer
        \item Unplug the ARIA and TA-DA power supplies 
        \item Remove and store the syringe or ring stand if applicable
        \item Rinse the funnel with acetone and dispose of the waste
        \item Tuck the end of the USB cord into the mounted section under the 
            edge of the hood so it does not present a tripping hazard
        \item Extract all data to the computer and appropriately rename them 
            (Refer to section ........)
        \item Shut down and unplug the tablet and the camera
        \item Remove the camera from the tripod using the plastic quick-release 
            lever and store the camera next to the lab computer
        \item Put all chemicals and syringes away in their proper places
        \item Remove any organic solid residue from all working surfaces 
            (See Section \ref{sec:spill_solid})
        \item Close all programs and shut down the computer
        \end{itemize}
    
    \item A hot furnace may be left in the open pressure vessel with the snorkel 
        venting it without waiting for it to cool
    \item Do not rinse out needles %% add policy for needles !!!!!!!
    \item Under normal use, disposable gloves may be thrown into the normal 
        trash receptacle instead of solid chemical waste
    \end{itemize}
    
\section{Data Extraction} 
During experiments data are being recorded on the lab computer, the datalogger 
and the camera. Both the camera and the datalogger on the TA-DA have SD cards 
with a 32 GB storage capacity that allows multiple runs to be recorded without 
extraction. The following policies are in place to ensure ease of use, 
efficiency and avoid common mistakes.
    \begin{itemize} 
    \item For the AIT setup, do not exceed 10 runs without extracting 
        temperature data to the computer and deleting the data from the SD  
        cards.
    \item All data should be extracted at least \textit{daily}
    \item Video data should be extracted and properly renamed as often as  
        possible (i.e. between every run or every other run) to ensure the 
        correct filenames are assigned to their corresponding video files
    
    \item File naming convention: 
        \begin{itemize}
        \item Filenames will be organized by the following values in order 
            separated by underscores ("\_")
                \begin{itemize}
                \item Compound name
                \item Phase of the compound ('g' for gases, 'l' for liquids,
                    's' for solids)
                \item Date of experiment with the format "YYMMDD"
                \item Time of day that data collection began for that run
                    using a 24 hour clock format "hhmm"
                \item Sample size in microliters (for liquids) or milligrams
                    (for solids and gases)
                \item Test temperature in degrees Celsius (rounded to the 
                    nearest integer)
                \end{itemize}
                
        \item For example: The filename for temperatures from an AIT experiment  
            where 100 microliters of liquid hexane were tested at 450
             \degree C on March 19, 2013 at 4:25 pm would be: \newline 
            "hexane\_l\_130319\_1625\_100\_450.csv"
        \end{itemize}

    \item Video and datalogger data must be processed (i.e. parsed, edited, 
        timestamped etc.) before being organized and therefore will be saved to 
        a different path initially

    \item After processing, all data should be organized according to the 
        following path convention:
        \begin{itemize}
        \item Path: \textit{C:\textbackslash Users\textbackslash 
            Public\textbackslash Documents\textbackslash AIT\textbackslash 
            data\textbackslash compound\_name\textbackslash filename.ext}
        \item All data, including videos, from the same run should have the same
            filename and path but different extensions except data from the 
            datalogger
        \item The datalogger filename convention should also have '\_dlog' at 
            the end of the name to distinguish it from the UI generated file 
            (e.g. "hexane\_l\_130319\_1625\_100\_450\_dlog.csv")
        \item When processing is finished all runs should have the following 
            4 files with the same name preceding them
            \begin{itemize}
            \item A .xlsx file (for temperature data with graphs and analysis)
            \item A .csv file (UI generated)
            \item A \_dlog.csv file (datalogger)
            \item A .avi/.mp4 file (video)
            \end{itemize}
        \end{itemize}

    \item The camera may be plugged in via USB and video extracted with
        GoPro\textsuperscript{\textcopyright} Quik software
        \begin{itemize}
        \item Connect the camera to the computer via a micro USB cable (See 
            Figure \ref{fig:cam_diag})
        \item Press the "info/wireless" button on the camera to connect the 
            camera to the computer
        \item Quik should be configured to open automatically extract video and 
            erase the microSD card when the camera connects to the computer
        \item Video files should be extracted to the DIPPR legacy server and 
            organized by date:
            \begin{itemize}
            \item Path: \textit{\textbackslash \textbackslash 
               dipprlegacy.et.byu.edu\textbackslash aitra\textbackslash 
               video\_import}
            \item Username: dipprleg\textbackslash aitra
            \item Password: hotflame16
            \end{itemize}
        \item If Quik is not configured to do this refer to the Quik manual for
            how to configure this (or ask me and I will configure it)
            \begin{itemize}
            \item \textit{C:\textbackslash Users\textbackslash 
            Public\textbackslash Documents\textbackslash AIT\textbackslash 
            docs\textbackslash GoPro\_App\_for\_Desktop\_User\_Manual.pdf}
            \end{itemize}
        \item Once extracted to the DIPPR legacy server, video data may be 
            timestamped and converted to .avi format on the server
        \end{itemize}
        
    \item To extract data from the datalogger
        \begin{itemize}
        \item Unplug the TA-DA from the computer
        \item Pull out the SD card from the datalogger and use the USB SD card 
            adapter to copy the"DATALOG.CSV" file into the "raw\_data" path and 
            rename it to the original filename with the date 
            tagged on in "YYMMMDD" format (e.g. "DATALOG\_130319.CSV")
            \begin{itemize}
            \item Path: \textit{C:\textbackslash Users\textbackslash 
            Public\textbackslash Documents\textbackslash AIT\textbackslash 
            data\textbackslash raw\_data}
            \end{itemize}
        \item Open the "DATALOG.CSV" file on the SD card, erase all data from
            it and save it, making sure to not change its name, extension or 
            file path
        \item Close all windows with the USB SD card adapter open (i.e. Excel 
            files, Windows Explorer etc.)
        \item Pull out the SD card without ejecting the unit from the computer
        \end{itemize}
    
    \item Ensure all files from the camera and datalogger are deleted after they
        have been properly saved in the data folder 
            
    \end{itemize}


\section{Spill Clean-up}
In the event of any spill, appropriate PPE specified in the corresponding SDS 
    should be used in clean-up. Always check the SDS for special considerations
    when cleaning up any compound.
    \subsection{Liquids}
    \begin{itemize}
    
    \item In the event of a small spill (i.e. less than 100 ml), the following 
        protocol should be followed:
        \begin{itemize}
        \item If the spill occurs in or out of the hood, use absorbent clay that
            can be found under the counter west of the sink to soak up the 
            bulk of the liquid and wipe up the rest with a paper towel
        \item Dispose of the clay, any disposable gloves and towels in the solid 
            waste container
        \end{itemize}    
    \item In the event of a large spill (i.e. greater than 100 ml), the 
        following protocol should be followed:
        \begin{itemize}
        \item If the spill occurs in the hood, use absorbent clay that can be 
            found in the lab to soak up the bulk of the liquid and wipe up the 
            rest with a paper towel
        \item Dispose of the clay, any disposable gloves and towels in the solid 
            waste container
        \item If the spill occurs outside the hood or the spill is particularly 
            large (e.g. an entire bottle of a flammable material breaks) 
            \textbf{perform the Emergency Shutdown Procedure (Section 
            \ref{sec:e_shtdn}), evacuate the lab and call: BYU Risk Management 
            and Safety - (801)-422-4468} 
        \end{itemize}
    
    \item Spills involving compounds that are particularly toxic or unstable 
        should always be considered large spills
    
    \end{itemize}

    \subsection{Solids} \label{sec:spill_solid}
    We will generally work with organic solids that readily dissolve
    in acetone. Researchers must always check chemical compatablity with acetone
    before dissolving any compound in acetone.
    \begin{itemize}
    \item Small amounts of organic solids may be dissolved in a small amount of 
        acetone and put in organic liquid waste
    \item Larger amounts of solids should be transferred to solid waste and the
        residue should be dissolved in acetone and discarded in liquid waste
    \end{itemize}


\section{Emergency Shutdown} \label{sec:e_shtdn}

    \begin{itemize}
    \item In the event of an emergency do the following:
        
        \begin{itemize}
        \item Close the air cylinder valve
        \item Power off the furnace
        \item Fully open the rotometer exhaust
        \item Unplug the furnace
        \item Stop the camera recording (if applicable)
        \item Shutdown and unplug the camera and tablet
        \item Unplug the ARIA and TADA
        \item Close all programs and shutdown the computer
        \end{itemize}
    
    \item If an emergency requires you to evacuate the lab, do only the first 
        2 steps
    \end{itemize}

\newpage	
\section{Experimental Setup and Maintenance}
    \subsection{Flask and Lid}
    \begin{itemize}
    \item \textbf{Latex or nitrile gloves and safety glasses are 
            required while working with the flask/lid assembly}
    \item The flask in the furnace must be exchanged for a clean flask in the 
        following situations:
        \begin{itemize}
        \item The next experiment will be for a different compound
        \item The next experiment will be for a new container of the same 
            compound
        \item There is reason to suspect that the flask has become 
            contaminated or substantially dirty
        \item The flask has been used for 10 runs without being cleaned
        \item Once the AIT has been found for a compound, the final measurements
            should be repeated with a clean flask to verify the results
        \end{itemize}
    
    \item Disassembling the Flask and Lid
        \begin{itemize}
        \item \textbf{The furnace may be too hot to open for several hours after
             an experiment}
        \item Unplug the thermocouples from the furnace
        \item Once the furnace is cool, remove flask/lid assembly
            \begin{itemize}
            \item Loosen (do NOT remove) the nut that secures the bracket and 
                the rubber hose to the top of the furnace with a wrench
            \item Move the bracket out of the way and remove Thermocouple 4 
                (along with the rubber hose) from the top of the furnace
            \item Move the mirror out of the way to allow the flask/lid assembly
                to come out
            \item Grip the assembly with both hands by the screws on top and 
                pull directly upward
            \item \textbf{NOTE: The flask/lid assembly is heavy and pulling it  
                out can be awkward. Please ask someone to help you remove it 
                if you are at all unsure about removing the assembly}
            \item The flask/lid assembly should easily come out of the furnace 
                without catching on anything
            \end{itemize}
        
        \item \textbf{Carefully} set the assembly on a table or other stable 
            surface with the flask on top (See Figure \ref{fig:f_lid_done})
        \item Ensure the bracket screw is loose
        \item Remove the circular spring from its groove and slide the ceramic
            halves of the lid apart sufficiently to allow the flask to be 
            removed
        \item Remove flask from lid assembly and  remove all of the aluminum 
            foil and thermocouples from the flask
        \item Discard the used aluminum foil in a normal trash can and set 
            aside the thermocouples in the hood or on a surface where they will
            not catch on anything or become damaged
        \item \textbf{Always store bulb flasks on the drying rack above the sink 
            or appropriately secured to a ring stand} (see "Flask Cleaning"
             below)
        \end{itemize}

\newpage      
    \item Assembling the Flask and Lid 
        \begin{itemize}
        \item Use the figures in this section as a reference when putting 
            together the assembly
        
        \item Use a \textbf{clean}, 500 ml, round bottom, long neck, bulb flask 
            (PYREX\textsuperscript{\textcopyright} 500mL Long Neck Boiling 
            Flask, Round Bottom, Tooled Mouth, Product No.: 4280-500 from 
            Corning Inc.)
        \item If dirty, wash out the flask using soap and water and dry as much 
            as possible (see "Flask Cleaning" below); be sure to rinse
            thoroughly
                \begin{itemize}
                \item Any leftover water will boil away when the furnace heats 
                up and before any measurements are taken
                \end{itemize}        
        \item Wrap entire flask in aluminum foil with thermocouples at the 
            bottom, side and top of the round part of the flask (thermocouples 
            should be touching the glass directly) (Refer to Figure 
            \ref{fig:wrap})
                \begin{itemize}
                \item NOTE: The more reflective side of the foil should always 
                    be facing inward
                \item Start by getting a long strip of aluminum foil (12" long 
                    or so)
                \item Use a utility knife to poke a hole near the middle of 
                    the foil and insert thermocouple 3 through the foil so the 
                    bead sits at the bottom of the flask and then wrap the foil 
                    around the bottom (1 and 2)
                \item Slide thermocouple 2 down to the approximate 
                    middle/equator of the flask between the flask and foil and 
                    use a second piece of foil to wrap further up the flask, 
                    ensuring the thermocouple wires run parallel up the side of 
                    the flask (3)
                \item Place thermocouple 1 at the top of the bulb of the flask 
                    (not on the neck of the flask) and use a third piece of foil
                     to wrap around the top starting at the middle (4)
                \item Add an additional layer of foil around  the flask so the 
                    wires are covered and run parallel when wrapping is finished
                    (5)
                \item Wrap additional foil around the neck of the flask to cover 
                    it completely and secure flask in lid assembly
                \item The thermocouple wires should emerge from the foil 
                    covering near the top (but not at the top) of the flask 
                    neck, allowing them to run between the two ceramic halves of
                    the lid assembly (6)
                \end{itemize}

\begin{figure}[H]
\centering
\includegraphics[width=1\textwidth]{wrap.jpg}
\caption{Steps for wrapping the flask in foil}
\label{fig:wrap}
\end{figure}

        \item Ensure the bracket screw is loose
        \item Fit the neck of the flask in the center hole of the ceramic lid 
            assembly with the lip of the flask fitting into the groove at the 
            base of the center hole on both sides
        \item Guide the thermocouple wires in the gap between the two ceramic 
            halves so they are out of the way when the flask/lid assembly is 
            inserted into the furnace
        \item Slide the loose half of the ceramic back in to be snug 
            around the flask neck, replace the spring, and tighten the nut on the top to hold it
            in position
                \begin{itemize}
                \item The two halves nearest to the top of the assembly should 
                    meet or very nearly meet; if they don't then some 
                    foil should be removed from the neck of the flask
                 \end{itemize}
        \item Use a circular spring to help hold the halves together
               
        
        \item Make a "donut" of foil wrapped around the neck of the flask that 
            will rest up against the bottom of the lid assembly
        \item Slide the foil "donut" up so and press it so it is flush against 
            the ceramic and restricts air flow around the opening
        \item Carefully turn the flask/lid assembly over making sure the flask 
            doesn't fall out
                \begin{itemize}
                \item \textbf{Do this over a table or close to a level surface 
                    to avoid accidental breaking of the flask}
                \item The flask will fit into the lid assembly somewhat loosely, 
                    but it shouldn't fall out
                \item If the flask falls out, remove it and add more foil 
                around the neck
                \end{itemize}
         
\begin{figure}[H]
\centering
\includegraphics[width=.45\textwidth]{insert_in_lid_w_donut.jpg}
\caption{Final state of the flask/lid assembly}
\label{fig:f_lid_done}
\end{figure}
        
        \item See Figure \ref{fig:f_lid_done} for the final flask/lid
            assembly before insertion into the furnace
        \item Place the prepared flask/lid assembly into the furnace by gripping
            the assembly with both hands by the screws on top and slowly 
            lowering the assembly into place
        \item Turn the flask/lid assembly so the thermocouple wires point away 
            from where researchers will be working
        \item Insert flask interior thermocouple (\#4) carefully down the flask 
            neck, making sure it goes straight in and the bead doesn't get 
            caught anywhere
                \begin{itemize}
                \item The bead of Thermocouple 4 should be suspended in the 
                    approximate center of the flask, not be touching any part
                \item The wire of Thermocouple 4 should run up the edge of the 
                    neck and not the middle to allow compound to be injected 
                    without making contact with the thermocouple
                \item Use the bracket on one of the two screws on top of the 
                    lid to secure the rubber hose holding the thermocouple
                    in place
                \item Tighten the nut on the bracket hand tight and then give a 
                    half turn with a wrench to secure the nut (See Figure 
                    \ref{fig:wrench_tight})
                \end{itemize}

\begin{figure}[H]
\centering
\includegraphics[width=.35\textwidth]{top_of_furnace_w_wrench.jpg}
\caption{Position thermocouple 4 with the rubber hose and tighten}
\label{fig:wrench_tight}
\end{figure}

        \item Connect the thermocouple connectors to join the  
            leads from the flask to the TA-DA
        \item Ensure the mirrors are set up correctly
        \item The final setup should resemble Figure \ref{fig:in_furnace}
        \end{itemize}
    
\begin{figure}[H]
\centering
\includegraphics[width=.65\textwidth]{Furnace_diagram_mod.jpg}
\caption{Diagram of the furnace when assembled}
\label{fig:in_furnace}
\end{figure}

    \item Flask Cleaning \newline
        For consistent experimental results, flasks must be as clean as possible
        (See Figure \ref{fig:clean_dirty}). Dirty flasks can terminate radical 
        reactions and artificially raise the AIT. To ensure flasks are as clean 
        as possible before use, the following steps are required for flask 
        cleaning:
        \begin{itemize}
		        \item Always begin by soaking the inside of the flask with soapy water 
            for 12 - 24 hours, regardless of how dirty it is
        \item While soaking, the flask should always be secured to a ring stand
        \item Wash out flask with soap and water, scrubbing the inside with 
            tube brushes
        \item For difficult stains, soak the flask inside with soapy water for 
            another 24 hours or longer if needed 
            \begin{itemize}
            \item During this process, scrub the inside and replace the 
                soapy water on a regular basis (generally every 12 - 24 hours)
            \end{itemize}
        
        \item Once all stains have been eradicated from the inside of the flask
            and the flask has been scrubbed in soapy water, rinse the inside and
            outside of the flask thoroughly
            \begin{itemize}
            \item Using hot water for rinsing is preferred but not required
            \item Rinse with tap water a minimum of 3 times, filling the flask
                with water, agitating the water for about 10 seconds, and then 
                dumping the water
            \item Repeat this process with distilled water available from the 
                smaller tap on the Northeast corner of the lab sink
            \end{itemize}
        
        \item If hard water spots or salt deposits appear on the inside of the 
            flask, rinse the inside of the flask with a small amount of vinegar
            to remove the deposits and repeat the rinse procedure above
        \item Once the flask has been cleaned and rinsed thoroughly, place the 
            clean flask on the drying rack over the sink
        \end{itemize}        
    \end{itemize}

\begin{figure}[H]
\centering
\includegraphics[width=.25\textwidth]{clean_dirty_flask.jpg}
\caption{A clean flask (dirty flask in the background)}
\label{fig:clean_dirty}
\end{figure}  

\newpage 
    \subsection{Furnace}
    	\subsubsection{Overview}
    \begin{itemize}
    \item The furnace, shown in Figure \ref{fig:furnace_pic}, is an encased 
        stack of ceramic insulation with cavities 
        cut out to allow space for the heating elements and the test flask
        (see Figure \ref{fig:in_furnace} for an internal diagram of the 
        furnace). The furnace is controlled with measurements taken at the 
        insulated furnace wall. This design causes the furnace to have
        large temperature gradients while in operation. As a result, the 
        setpoint temperature and the flask temperature will almost always 
        differ significantly (as much as 25 K in some cases). Therefore, 
        setpoints must be chosen between approximately 10 - 20 K above the 
        desired temperature to reach that temperature inside the flask.
        \textbf{The reported AIT must be taken from the internal flask 
        temperature (Thermocouple 4) and NOT the control thermocouple inside
        the furnace}
    
    \begin{figure}[H]
    \centering
    \includegraphics[width=.45\textwidth]{Furnace_pic_diagram.jpg}
    \caption{AIT Furnace}
    \label{fig:furnace_pic}
    \end{figure}
    
    \item When powered on initially, the furnace may take up to 2 hours or more  
        to reach a desired temperature and thermally equilibrate
    \item Any time a desired temperature is reached, allow at least 30 
        minutes for thorough thermal equilibration in the flask; allow extra 
        time during initial start up
     
     \end{itemize}

\subsubsection{Furnace Operation (See Figure \ref{fig:furnace_pic})}
      \begin{itemize}
      \item Power on the furnace with the power switch and use the temperature
            controller to choose a setpoint temperature
        \begin{itemize}
            \item To change the set point, press the up or down     arrows until 
                the desired temperature is reached
            \item The lower (green) display is the setpoint and     the upper 
                (red) display is the control thermocouple temperature
        \end{itemize}
        
        \item When shutting down, turn off the power switch and unplug the furnace
        \end{itemize}

        
    \subsection{Camera and Tablet} \label{sec:cam_tab}
    \subsubsection{Overview}
    	\begin{itemize}
    	\item Prior to using the experimental setup, all researchers must become 
        familiar with basic use and operation of the 
        GoPro\textsuperscript{\textcopyright} HERO4
        Session\textsuperscript{TM} camera and the Samsung Galaxy Tab A Tablet. 
        
        More detailed instructions on how to do basic tasks may be found at the 
        following URLs:
        \begin{itemize}
        \item \texttt{https://shop.gopro.com/softwareandapp}
        \item \texttt{https://gopro.com/help/articles/Block/How-to-Pair-the-Camera-with-the-GoPro-App\#HERO4 Session}
        \item \texttt{https://gopro.com/help/articles/Block/Getting-Started-with-the-GoPro-App}
        \item \texttt{http://www.samsung.com/us/support/owners/product/galaxy-tab-a-8-0-wi-fi}
        \end{itemize}
    
    
\begin{figure}[H]
\centering
\includegraphics[width=.6\textwidth]{Camera_diagram.jpg}
\caption{GoPro\textsuperscript{\textcopyright} HERO4
        Session\textsuperscript{TM} Camera Parts}
\label{fig:cam_diag}
\end{figure}

\begin{figure}[H]
\centering
\includegraphics[width=1\textwidth]{tablet.jpg}
\caption{Samsung Galaxy Tab A}
\label{fig:tablet}
\end{figure}
    
    \item Refer to Figures \ref{fig:cam_diag} and \ref{fig:tablet} for camera 
        and tablet setup
        
     \end{itemize}
        
       
    \subsubsection{Connecting to the camera} 
        \begin{itemize}
        \item Firmly press and release the "info/wireless" button on the back of the camera multiple times until 
            you see "APP & RC" on the camera status screen
        \item Press the "shutter/select" button to confirm your selection
            \begin{itemize}
            \item The "wireless status" (blue) light will begin flashing. This 
                indicates the camera is broadcasting a Wi-Fi signal 
            \end{itemize}
        \item Power on the tablet by holding down the power key (top, right side) until you see the
            splash screen indicating the tablet is booting up
        \item Once the tablet has booted, swipe to get to the home screen and 
            select the "Settings" app
        \item Select the Wi-Fi settings at the top of the list on the left side 
            of the screen
        \item Select the Wi-Fi network labelled "ait\_cam\_2016" then select 
            "connect" on the message box that pops up (FYI: the wifi password is
            "hotflame16")
        \item Once the tablet has connected to the Wi-Fi, return to the home 
            screen by pressing the home key
        \item Open the GoPro Capture App (app is labelled "Capture" on the home 
            screen) 
        \item Select the connect box on the top left corner of the screen to connect to camera
        \item Press the camera icon in the center of the screen
            \begin{itemize}
            \item The camera will make a beeping noise and the camera view will 
                open on the tablet
            \end{itemize}
        \end{itemize}
    
    \subsubsection{Shutdown}
        \begin{itemize}
        \item To shutdown the camera:
            \begin{itemize}
            \item Press the "info/wireless" button 
                until the camera status screen reads "Turn Wi-Fi Off"
            \item Press the "shutter/select" button to confirm your selection
                \begin{itemize}
                \item The "wireless status" (blue) light will stop flashing
                \end{itemize}
            
            \item Press the "info/wireless" button until the camera status 
                screen reads "Exit"
            \item Press the "shutter/select" button to confirm your selection
                \begin{itemize}
                \item The camera will shutdown 
                \end{itemize}
            
            \end{itemize}
        
        \item To shutdown the tablet:
            \begin{itemize}
            \item Press the "Recent" button to bring up all opened programs and 
                close all programs by swiping on them or pressing the 'X' in the
                top right corner
            \item Press and hold the Power key until the option to power of 
                pops up then press power off
                \begin{itemize}
                \item The tablet will shutdown
                \end{itemize}
           
            \end{itemize}
        
        \end{itemize}
    
    \subsubsection{Other Information}
    	\begin{itemize}
        \item Camera Operation
        	\begin{itemize}
        	\item All operations may be done remotely on the tablet via Wi-Fi or 
            directly with the "info/wireless" and "shutter/select" buttons on 
            the camera. For experimental purposes, only basic operations
            will be covered. For more detail on camera 
            operation please see the URLs above
        	\item In the camera's off or normal modes the "shutter/select" button 
            toggles recording or standby; the camera will automatically shut off
            after a few seconds on standby
        	\item If the camera is remotely controlled, the on screen red button 
            toggles recording or standby
        	\item During recording, the camera will not allow viewing via the 
            tablet. This is due to the high frame rate of our experiments
        	\item Captured video may be reviewed and managed remotely with the grid
            button on the bottom left corner of the screen
        	\item The camera may be powered on and off remotely with the power  
            button on the top right corner of the screen. The camera should be 
            powered off between experiments or when not in use
        	\end{itemize}
        
        
        
    	\item Batteries:
        	\begin{itemize}
        	\item Recharging power supplies and USB cables are available for both
            the tablet and camera
        	\item Both the camera and the tablet may be charged while in use
        	\item Do NOT charge tablet with the computer as it does not deliver 
            enough current for effective charging
        	\item Batteries should be allowed to discharge to between 10 - 20\%
            before recharging
        	\item Batteries should always be recharged to 100\% capacity before 
            unplugging
        	\item Do not overcharge any battery. Do not leave any battery 
            charging overnight
        	\end{itemize}
        
    	\end{itemize}

\end{document}
